%%%%%%%%%%%%%%%%%%%%%%%%%%%%%%%%%%%%%%%%%%%%%%%%%%%%%%%%%%%
%%  SONGBOOK style
%%  
%%  actualy provides only makro \inputsong
%%  which reads and parse one songfile
%%
\newtoks\savenum
\newbox\lasttext
\newbox\lastchord
% define some spec. tokens
\def\SharpChar{"}%
\def\FlatChar{b}%
\def\unspace{\vrule width 0pt}
%
\def\inputsong#1{%
  \vbox\bgroup
  \begingroup
  \catcode`\^^M=13
  \expandafter\getheader\input #1 
  \EOLtok\EOLtok\ENDSONG% make sure that song will end during skipEOLs
}
\def\ENDSONG{
  \unskip\endgroup\egroup
  \vskip\InterSongSkip
  }%
%%%%%%%%%%%%%%%%%%%%%%%%%%%%%%%%%%%%%%%%%%%%%%%%%%%%%%%%%%
%% header
\def\({(}\def\){)}
{\catcode`\(=13\catcode`\^^M=13
% getheader - reads 5 header fields, call formating macro, 
\gdef\getheader N:#1^^MA:#2^^MAC:#3^^MZ:#4^^MZC:#5^^M{%
   \MakeSongHeader{#1}{#2}{#3}{#4}{#5}%
   \beginsong}%
% beginsong - setup environment
\gdef^^M{\firstEOL}%
\global\let\EOLtok=^^M% active EOL token
\gdef\beginsong{%
  \catcode`\�=0%
  \catcode`\(=13\def({\beginchord}% 
  \skipEOLs}% skip empty lines and start new par
}%
%%%%%%%%%%%%%%%%%%%%%%%%%%%%%%%%%%%%%%%%%%%%%%%%%%%%%%
%% text reading
\def\begintext{%
  \ifvmode \noindent \MakeParBegin \fi%
  \setbox\lasttext=\hbox\bgroup\aftergroup\endtext%
  \unspace}% to eat possible space
\def\endtext{%
  \ifvoid\lastchord\box\lasttext% test trivial case
  \else\typechord%
  \fi}%
%%%%%%%%%%%%%%%%%%%%%%%%%%%%%%%%%%%%%%%%%%%%
%% line/par. govering - we have to use futurelet as TeX, can't
%                       read parameters over file end. Sigh!

% firstEOL - test of end of par
\def\firstEOL{\futurelet\nextTOK\firstEOLx}%
\long\def\firstEOLx{%
  \ifx\nextTOK\EOLtok% end par
     \hfil\egroup\par% close text environment and par.
     \let\next=\skipEOLs%
  \else%
     \hfil\egroup% close text environment
     \hfill\break% hard line break
     \MakeLineIndent%
     \let\next=\begintext% start new line
  \fi%
  \next}%

% skipEOLs - skip empty lines, manage end of SONG
\def\skipEOLs{\futurelet\nextTOK\skipEOLsA}
\def\skipEOLsA{%
  \ifx\nextTOK\EOLtok\let\next=\skipEOLsB%
  \else\ifx\nextTOK\ENDSONG\let\next=\relax%
  \else% start read of par type
     \let\XTOKnext=\getverse%
     \let\next=\readXTOK%
  \fi\fi\next}%
\def\skipEOLsB#1{\skipEOLs}

% readXTOK - automata kernel
\def\readXTOK#1{%
  \XTOKnext#1%
  \ifx\XTOKsave\unspace% do not add : if we read par type sucessfully
  \else\expandafter\gdef\expandafter\XTOKsave\expandafter{\XTOKsave#1}%
  \fi%
  \ifx\XTOKnext\begintext\let\next=\BEGINTEXT% start new par
  \else\let\next=\readXTOK% continue read par type
  \fi\next}
\def\BEGINTEXT{\begintext\XTOKsave}
%%%%  
%%%%  
\def\getverse#1{%
  \global\let\XTOKsave=\empty%
  \let\XTOKnext=\getnum%
  \ifx R#1\global\let\TypeOfVerse=R% read next tok
  \else\ifx M#1\global\let\TypeOfVerse=M% read next tok
    \else\ifx C#1\global\let\TypeOfVerse=C%
      \else\global\let\TypeOfVerse=V\getnum#1% continue with current
  \fi\fi\fi}%
%%%%  
\def\getnum#1{%
  \let\XTOKnext=\relax%
  \ifdigit{#1}{\let\XTOKnext=\getnum}%
  \ifx\XTOKnext\relax% end of the number 
    \ifx:#1%
      \ifx\TypeOfVerse V\xdef\NumOfVerse{\XTOKsave}%
      \else\xdef\NumOfVerse{\let\noexpand\trash=\XTOKsave}%
      \fi%
      \let\XTOKnext=\begintext%
      \global\let\XTOKsave=\unspace% skip optional space after :
    \else% Oh, it is no verse mark, return it 
      \global\let\TypeOfVerse=X%
      \let\XTOKnext=\begintext%
    \fi%  
  \fi}%
%%%%
\def\ifdigit#1#2{% it's so long as we can not get ASCII num of \nextChar
 \ONEtest0#1{#2}\ONEtest1#1{#2}\ONEtest2#1{#2}\ONEtest3#1{#2}\ONEtest4#1{#2}%
 \ONEtest5#1{#2}\ONEtest6#1{#2}\ONEtest7#1{#2}\ONEtest8#1{#2}\ONEtest9#1{#2}}%
\def\ONEtest#1#2#3{\ifx#1#2#3\fi}%
%%%%%%%%%%%%%%%%%%%%%%%%%%%%%%%%%%%%%%%%%%%%%%%%%%%%
% "chord environment" 
% 
% chord has a form:
% [A-H][b"]?\(<modifier>\)?
% \((<general exp>)\)?\(/?<bass>\)?
% \|\(<restriced exp>\)?\(/<bass>\)?
% <modifier> .. arbitrary text, but meaningfull are: +,-,m,mi,maj,dim,sus,add
% <general exp> .. arbitraty balanced text
% <restricted exp> .. arbitrary text without slash
%
% you can write more chords separated by a space
% 
% examples:
% (A) (A") (Ab) (Am) (A7) (A6+) ... produce what you expect 
% ((A)) optional chord 
% (A((4))) optional additional 4
% (A(6/9)/E) complex exp. with bass suggestion
% (C G C) more then one chord over one nible
%
% implementation: finite automata with support for paired ()
% - all chords are recorded into allchords.aux
%%%%%%%%%%%%%%%%%%%%%%%%%%%%%%%%%%%%%%%%%%%%%%%
\newwrite\logchords%
\immediate\openout\logchords=allchords.aux
\def\logchord{%
  \immediate\write\logchords{\savechord}%
  \global\let\savechord=\empty}%
% readchord - automata kernel
\def\readchord#1{%
  \expandafter\gdef\expandafter\savechord\expandafter{\savechord#1}% 
  % hack for more chords separated by the space
  \ifx\chspace#1\logchord\readchordd{)}% empty savechord -> chord ended by space
  \else\readchordd{#1}\fi% save token and do chordnext
  \ifx\chordnext\ENDAUTOMATA \let\next=\relax% .. so stop reading 
  \else\let\next=\readchord% continue
  \fi\next}
\def\readchordd#1{%
  \ifx\CHegroupchar#1\egroup% end of pair, must have \aftergroup !!
  \else \chordnext{#1}\fi}%
%
% testtok - to test various tokens without many included "if"s
\def\testtok#1#2#3{%
  \ifx#1#2\ifx\chordnext\relax\let\chordnext=#3\fi%
  \fi}%
% main pair 
\def\beginchord{%
  \egroup% end text environment
  \setbox\lastchord=\hbox\bgroup$% 
  \catcode`\^^M=15\catcode`\(=12%
  \setchspace
  \expandafter\let\expandafter\sharptok\SharpChar
  \expandafter\let\expandafter\flattok\FlatChar
  \bgroup\aftergroup\endchord
  \bchord\readchord}%
{\catcode`\ =12\gdef\setchspace{\catcode`\ =12\let\chspace= }}%
\def\endchord{%
  \unkern$\egroup% end math chord box 
  \let\chordnext=\ENDAUTOMATA%
  \begintext}%
% one chord group
\def\bchord{%
  \bgroup\aftergroup\echord%
  \global\let\savechord=\empty%
  \let\CHegroupchar=)\let\chordnext=\bCHopt}%
\def\echord{%
  \ifx\savechord\empty% chord ended by space
    \dimen0=\fixchordspace\advance\dimen0 by \pluschordspace\hskip\dimen0%
    \let\next=\bchord%
  \else%                chord ended by ) -> end whole chord env.
    \logchord\let\next=\egroup\fi\next}%
%  
% opt. chord pair
\def\bCHopt#1{%
  \ifx(#1 \left(\bgroup\aftergroup\eCHopt%
    \let\chordnext=\bCHbase%
  \else\bCHbase{#1}%
  \fi}%
\def\eCHopt{\right)\let\chordnext=\CHerrend}%
% error if there is something else than ) after opt chord )
\def\CHerrend#1{\errmessage{Wrong end of the chord}\egroup#1}%
%
% chord base 
\def\bCHbase#1{\setformatA\MakeBase{#1}\let\chordnext=\eCHbase}%
\def\eCHbase#1{\let\chordnext=\bCHtype\CHreadtone{#1}}%
% CHreadtone - for base and bass tone modifiers
\def\CHreadtone#1{%
  \ifx#1\sharptok\MakeSharp
  \else\ifx#1\flattok\MakeFlat
  \else\chordnext{#1}%
  \fi\fi}  
%
% chord type - such text as m,mi,maj,sus,+,-,dim
\def\bCHtype#1{\CHtype{#1}}% this is now obsolate, but wait ...
\def\CHtype#1{%
  \let\chordnext=\relax%
  \testtok(#1\bCHgeneralexp% normal exp in ()
  \testtok/#1{\bCHbass\unkern}%
  \testtok+#1{\CHtype\unkern\MakeBasePlus}%
  \testtok-#1{\CHtype\unkern\MakeBaseMinus}%
  % simple exp begin with digit and end with / 
  \testtok2#1{\relax\bCHrestrictexp{#1}}% 
  \testtok4#1{\relax\bCHrestrictexp{#1}}% 
  \testtok6#1{\relax\bCHrestrictexp{#1}}% 
  \testtok7#1{\relax\bCHrestrictexp{#1}}% 
  \testtok9#1{\relax\bCHrestrictexp{#1}}% 
  % very rare cases
  \testtok1#1{\relax\bCHrestrictexp{#1}}% 
  \testtok3#1{\relax\bCHrestrictexp{#1}}% 
  \testtok5#1{\relax\bCHrestrictexp{#1}}% 
  %hack for write (A0(4)) and no (A((4)))
  \testtok0#1\bCHrestrictexp% 
  \testtok!!{\CHtype\unkern#1}% normaly type all text
}  
%
% exp pairs
\def\bCHgeneralexp#1{^\bgroup\aftergroup\eCHexp\setformatB\CHexp{#1}}%
\def\bCHrestrictexp#1{^\bgroup\aftergroup\eCHexp\setformatB
  \let\CHegroupchar=/\CHexp{#1}}%
\def\eCHexp{\MakeAfterExpKern\let\chordnext=\bCHbass}%
%
\def\CHexp#1{%
  \let\chordnext=\relax%
  \testtok+#1{\CHexp\MakeExpPlus}%
  \testtok-#1{\CHexp\MakeExpMinus}%
  \testtok(#1{\CHexp\bCHoptexp}% optional exp.
  % if we have ) here it must be end of whole chord 
  % -> override normal rules
  \testtok)#1{\relax\egroup\egroup}% end exp and main/opt
  \testtok!!{\CHexp#1}% normal type
}%
\def\bCHoptexp{(\bgroup\aftergroup\eCHoptexp\let\CHegroupchar=)}%
\def\eCHoptexp{\noexpand)}%
%
% begin bass section only if there is nontrivial material
% using _ for raise, formula for subscript raise is:
% max(d(X)+s_19,s_16,h(X)-4/5abs(s_5)), for atom X
\def\bCHbass#1{\ifx/#1 
  \else\MakeSlash%
  \fontdimen5\textfont2=2ex
  \fontdimen19\textfont2=-2ex
  \fontdimen16\textfont2=\MakeBassRaise%
  _\bgroup\aftergroup\egroup% after bass group close chord
  \textstyle\setformatC\MakeBase{#1}%
  \let\CHegroupchar=\relax% take control over ")"
  \let\chordnext=\eCHbass\fi}%
\def\eCHbass#1{\ifx)#1\egroup% end of bass and chord too
  \else\let\CHegroupchar=)\let\chordnext=\CHerrend\CHreadtone{#1}\fi}% 
%%%%%%%%%%%%%%%%%%%%%%%%%%%%%%%%%%%%%%%%%%%%%%%%%%%%%%%
%% this is the end of the parsing macros
%%%%%%%%%%%%%%%%%%%%%%%%%%%%%%%%%%%%%%%%%%%%%%%%%%%%%%%
%% Typeseting macros follows, many parameters can be changed  
%%%%%%%%%%%%%%%%%%%%%%%%%%%%%%%%%%%%%%%%%%%%%%
% typechord - put boxes \lastchord and \lasttext together
\def\fixchordspace{0.5em}  % min space between chords 
\def\pluschordspace{0.2em} % additional space, if we do prolongation of text
\def\chordraise{2.1ex}     % raise of the chord
%%%%%%%%%%%%%%%%%%%%%%%%%%%
% prolongation of the text inside of word
\def\chordinshypen{%
  \setbox0=\hbox{-}%
  \advance\dimen0 by -\wd0\divide\dimen0 by 2%
  \ifdim\dimen0<0.3em\dimen0=0.3em\else\fi%
  \hskip\dimen0{-}\hskip\dimen0%
}  
%%%%%%%%%%%%%%%%%%%%%%%%%%
% prolongation of the text in space
\def\chordinsspace{%
  \unskip%
  \ifdim\dimen0>3em{%
    \advance\dimen0 by -1.5em%
    \hskip.75em\leaders\hbox{--}\hskip\dimen0 plus 1em\hskip.75em%
  }%
  \else{\hskip\dimen0 plus 1em}%
  \fi%
}%
%%%%%%%%%%%%%%%%%%%%%%%%%%%
\def\typechord{%
  \dimen0=\wd\lastchord% d0=(d1=wd(chord)+fixchordspace)-wd(text)
  \advance\dimen0 by \fixchordspace
  \dimen1=\dimen0 
  \advance\dimen0 by -\wd\lasttext
  \raise\chordraise\rlap{\unhbox\lastchord}% insert chord with zero width
  % test if text box is just space
  \setbox1=\copy\lasttext\setbox0=\hbox{\unhbox1\unskip\unskip\unskip}%
  \ifdim\wd0=0pt\setbox\lasttext=\hbox{\hskip\dimen1}\dimen0=-1pt\fi% 
  \ifdim\dimen0<0pt{\unhbox\lasttext}% the best case 
  \else% try streach the text
    \setbox0=\hbox to \dimen1{\unhbox\lasttext}%
    \ifnum\badness<250\box0% still good case
    \else\unhbox0% fill the space or hypen a word      
       \advance\dimen0 by \pluschordspace
       \ifdim\lastskip=0.0pt\chordinshypen%
       \else\chordinsspace\fi%
    \fi%
  \fi%
}  
%%%%%%%%%%%%%%%%%%%%%%%%%%%%%%%%%%%%%%%%%%%%%
% Chord typeseting
%
%% MakeBase type base letter with possible \kern for supperscripts after
\def\MakeBase#1{\csname MakeTone#1\endcsname}
\def\MakeToneA{A\kern-.15em}
\def\MakeToneB{B}
\def\MakeToneC{C}
\def\MakeToneD{D}
\def\MakeToneE{E}
\def\MakeToneF{F}
\def\MakeToneG{G}
\def\MakeToneH{H}
%% how to type special characters
\def\MakeScrut{{\setbox0=\hbox{H}\vbox to \ht0{}}}
\def\MakeFlat{^\flat\MakeScrut}
\def\MakeSharp{^\sharp\MakeScrut}
\def\MakeSlash{\char`/}
\def\MakeExpPlus{{+}}
\def\MakeExpMinus{{-}}
\def\MakeBasePlus{\raise+.2ex\hbox{+}}
\def\MakeBaseMinus{{\scriptstyle -}}
%% some kerns and raises
\def\MakeAfterExpKern{\kern-.1em}
\def\MakeBassRaise{-.5ex}
%% set fonts for A=base tone and chord type
%%               B= exponent  C= bass tone
% useable fonts:
% qhvb - helvetica bold, PS, no sizes 
% qhvcb - helvetica condensed bold, PS, no sizes
% ebbx10 - helvetica bold
% cmbrbx10 - hlv. bld.
% ebsr10 - hlv.
% pnssb10 - pandora bld. non. sym
% t1xbss, tyxbss - hlv. bld. PS, no sizes
% vnssdc10 - round hlv. cond. bld.      
% rpagd, rphvb,rphvbrn
\newfam\ChordFam
\font\ChordFontText=qhvb at 8pt
\font\ChordFontExp=qhvb at 5pt
\font\ChordFontBass=qhvr at 8pt
\font\CHSymFont=cmmib10 at 8pt
\font\SongNameFont=csbx12 at 14pt
\font\AuthorFont=csti10
\font\TextFont=csr10 at 11pt
\font\PageFont=csr9
%
\textfont\ChordFam=\ChordFontText
\scriptfont\ChordFam=\ChordFontExp
\scriptscriptfont\ChordFam=\ChordFontBass
\def\setformatA{\fam=\ChordFam\textstyle} 
\def\setformatB{\fam=\ChordFam\scriptstyle}
\def\setformatC{\fam=\ChordFam\scriptscriptstyle}
%%%%%%%%%%%%%%%%%%%%%%%%%%%%%%%%%%%%%%%%%%%%%%%%%%%%5
%% text typeseting
%%
%% how to make song header from parametres:
%% name,author,author code,source, source code
\def\MakeSongHeader#1#2#3#4#5{{%
\noindent\SongNameFont\unspace#1\hfill\unspace#5\par%
\if:#2:\else\AuthorFont\unspace#2\par\fi}}
%
%% how to begin special paragraphs
%% every paragraph in the song environment is started by
%% \noindent so the user can setup indentation of one's own
\def\MakeParBegin{%
  \ifx V\TypeOfVerse \hbox to \parindent {\MakeVerse\hfil}%
  \else\ifx R\TypeOfVerse \hbox to \parindent {\MakeRef\hfil}%
  \else\ifx C\TypeOfVerse \hbox to \parindent {Rec:\hfil}
  \else\ifx M\TypeOfVerse% 
  \else\hbox{\hskip\parindent}%
  \fi\fi\fi\fi}
\def\MakeVerse{\NumOfVerse.}
\def\MakeRef{R\NumOfVerse:}
%
%% this makes line indentation
\def\MakeLineIndent{%
  \ifx M\TypeOfVerse%
  \else\hbox{\hskip\parindent}\fi%
}  
%
\newskip\InterSongSkip
\InterSongSkip=12pt plus 240pt
%
\def\MakeVerse{\NumOfVerse.}
\def\MakeRef{R\NumOfVerse:}
\parindent=20pt
\parskip=1.2ex plus 2ex
% page layout
\newdimen\InnerMargin\InnerMargin=1.5cm
\newdimen\OuterMargin\OuterMargin=1cm
\newdimen\TopMargin\TopMargin=1cm
\newdimen\BottomMargin\BottomMargin=1cm
\def\vPageSize{21cm}
\def\hPageSize{148mm}
%%%%%%%%%%%%%%%%%%%%%%%%%%%%%%%%%%%%%%%%%%%%
% Formating of pages - output rutine
%%%%%%%%%%%%%%%%%%%%%%%%%%%%%%%%%%%%%%%%
\output={\lyricoutput}
\def\lyricoutput{\poisepages
  \hoffset=-1in
  %left page
  {\advance\hoffset by \OuterMargin
   \shipout\vbox{\vskip\TopMargin\box1\MakePageNum}}
  %right page 
  {\advance\hoffset by \InnerMargin
   \shipout\vbox{\vskip\TopMargin\box2\MakePageNum}}
}
\def\MakePageNum{\global\advance\pagenum by 1
  \PageFont\vskip 2ex
  \hbox to \hsize{\PageFont
    \ifodd\pagenum\the\pagenum\hfill
    \else\hfill\the\pagenum\fi
  }}
\def\measure{\vbox to 0pt{\hbox to 0pt{\vrule height 18cm}}}
  % iterativne se budu snazit najit strankovy zlom dany prirozenou
  % vyskou prvni strany \firstvsize
  % 0) \firstvsize=0.5\pagetotal(obou stran)
  % 1) odlomim z \box255 to \firstvsize
  % 2) odlomene i zbytek natahnu na \vsize
  % 3) jestli se od posledka nezmenil soucet badness tak koncim
  % 4) kdyz ma 1. str vetsi badness tak zvetsim \firstvsize jinak zmensim
\def\StepSize{1pt} % abs iteration step
\def\poisepages{\rmvboxes
  % \dimen0 - try vsize of the first page
  % \dimen1 - iteration step of \dimen0
  % \count1,2 - badness of the left,right page
  % \count3,0 - actual,last \count1+\count2
  %
  \setbox0=\vbox{\unvcopy255}\dimen0=0.5\ht0% aproximate one page height
  \ifdim\vsize<\ht0\message{Page overfull (\the\vsize, \the\ht0) !!}\fi
  \setbox0=\box255% save copy
  \setbox255=\copy0\splitpages% first try 
  \ifnum\count1>\count2\dimen1=\StepSize\fi
  \ifnum\count1<\count2\dimen1=-\StepSize\fi
  \let\cycle=\loop
  \loop
     \advance\dimen0 by \dimen1
     \count0=\count3\setbox255=\copy0\splitpages% try next height
     \message{d: \the\dimen0, \the\dimen1
              b: \the\count1, \the\count2, \the\count0, \the\count3}%
     \ifnum\count3>\count0\let\cycle=\relax\fi
  \ifx\cycle\loop
  \repeat
  % revert last try
  \setbox255=\copy0\advance\dimen0 by -\dimen1\splitpages
  }
\def\rmvboxes{% remove song vbox envelps to allow intersong page break
  \setbox1=\vbox{\unvbox255
  \loop
     \unskip\unskip\setbox0=\lastbox
     \ifvbox0\global\setbox255=\vbox{\unvbox0\vskip\InterSongSkip\unvbox255}%
  \repeat}
  \ifdim\ht1>0pt\showboxbreadth=100\showboxdepth=1\showbox1\fi}
\def\splitpages{%
%  \showboxbreadth=200\showboxdepth=1\showbox255
  \setbox1=\vsplit255 to \dimen0
%  \showboxbreadth=100\showboxdepth=1\showbox255
  \setbox1=\vbox to 0.5\vsize{\unvbox1}\count1=\badness
  \setbox2=\vbox to 0.5\vsize{\unvbox255}\count2=\badness
  \ifnum\count1=\count2
    \setbox3=\vbox{\unvcopy1}
    \setbox4=\vbox{\unvcopy2}
    \ifdim\ht3<\ht4\count1=0\count2=1
    \else\count1=1\count2=0\fi
  \fi  
  \count3=\count1\advance\count3 by \count2%
}  
\newcount\pagenum
% vertical setting
\vsize=\vPageSize 
\advance\vsize by -\TopMargin
\advance\vsize by -\BottomMargin
\setbox0=\vbox{\MakePageNum}
\advance\vsize by \ht0
\pagenum=0
\vsize=2\vsize% vzdy formatuju 2 strany najednou
\voffset=-1in
\advance\voffset by \TopMargin
% horizontal setting
\hoffset=\hPageSize
\advance\hsize by -\InnerMargin
\advance\hsize by -\OuterMargin
%




\tracingmacros=2
\tracingcommands=2
%\tracingpages=1
%\tracingrestores=1
%\tracingoutput=5

%Testovaci text s akordy
%Kaz(Abm(7+)/C#)dykonecradku(Dm(7)/Bb)je ... 
%(D+)du(Am)le(B-)zity
%
%... i prazdny
%(Cm)(C+)(C-)(Cdim)(Cb)(C#)
%(Dm)(D+)(D-)(Ddim)(Db)(D#)
%(Em)(E+)(E-)(Edim)(Eb)(E#)
%(Fm)(F+)(F-)(Fdim)(Fb)(F#)
%(Gm)(G+)(G-)(Gdim)(Gb)(G#)
%(Am)(A+)(A-)(Adim)(Ab)(A#)
%(Hm)(H+)(H-)(Hdim)(Hb)(H#)
%(Bm)(B+)(B-)(Bdim)(Bb)(B#)
%


%\inputsong{Amazonka.sng}
%\bye
