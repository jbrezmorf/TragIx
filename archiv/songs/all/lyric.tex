%%%%%%%%%%%%%%%%%%%%%%%%%%%%%%%%%%%%%%%%%%%%%%%%%%%%%%%%%%%
%%  SONGBOOK style
%%  
%%  actualy provides only makro \inputsong
%%  which reads and parse one songfile
%%
\newtoks\savenum
\newbox\lasttext
\newbox\lastchord
% define some spec. tokens
\def\SharpChar{"}%
\def\FlatChar{b}%
\def\unspace{\vrule width 0pt}
%
\def\inputsong#1{%
  \vbox\bgroup
  \begingroup
  \catcode`\^^M=13
  \expandafter\getheader\input #1 
  \ENDSONG
}
\def\ENDSONG{
  \unskip\endgroup\egroup
  \vskip\InterSongSkip
  }%
%%%%%%%%%%%%%%%%%%%%%%%%%%%%%%%%%%%%%%%%%%%%%%%%%%%%%%%%%%
%% header
\def\({(}\def\){)}
{\catcode`\(=13\catcode`\^^M=13
% getheader - reads 5 header fields, call formating macro, 
\gdef\getheader N:#1^^MA:#2^^MAC:#3^^MZ:#4^^MZC:#5^^M{%
   \MakeSongHeader{#1}{#2}{#3}{#4}{#5}%
   \beginsong}%
% beginsong - setup environment
\gdef\beginsong{%
  \def^^M{\doEOL}\let\EOLtok=^^M % to have exactly same token for testing
  \catcode`\�=0%
  \catcode`\(=13\def({\beginchord}% 
  \skipEOLs\relax}% do not eat first char
  % skip all ^^M and begin new paragraph
}%
%%%%%%%%%%%%%%%%%%%%%%%%%%%%%%%%%%%%%%%%%%%%%%%%%%%%%%
%% text reading
\def\begintext{%
  \ifvmode \noindent \MakeParBegin \fi%
  \setbox\lasttext=\hbox\bgroup\aftergroup\endtext%
  \unspace}% to eat possible space
\def\endtext{%
  \ifvoid\lastchord\unhbox\lasttext% test trivial case
  \else\typechord%
  \fi}%
%%%%%%%%%%%%%%%%%%%%%%%%%%%%%%%%%%%%%%%%%%%%
%% line/par. govering - another finite automata with \futurelet
%% we use two \hfil at the end of line, because first could
%% be closed in the last \lasttext box, if there is a chord
\def\doEOL{\futurelet\nextEOL\firstEOL}%
\long\def\firstEOL{%
  \ifx\nextEOL\EOLtok% end par
     \hfil\egroup\hfil\par% close text environment and par.
     \penalty\InterVersePenalty%
     \let\next=\skipEOLs% skip EOLs and begin new one
  \else\ifx\nextEOL\ENDSONG% end of line inside a paragraph
     \hfil\egroup\hfil\par
     \let\next=\relax
  \else   
    \hfil\egroup\hfil% close text environment
    \break% hard line break
    \MakeLineIndent%
    \let\next=\begintext% start new line
  \fi\fi\next}%
%%%%  
\def\skipEOLs#1{\futurelet\Xtok\skipEOLsX}
\def\skipEOLsX{%
  \ifx\Xtok\EOLtok\let\next=\skipEOLs%
  \else\ifx\Xtok\ENDSONG\let\next=\relax%
  \else\let\next=\getverse%
  \fi\fi\next}%
%%%%  
\def\getverse{%
  \let\next=\getnum\global\savenum={}\xdef\SAVEDTEXT{}%
  \ifx R\Xtok\gdef\TypeOfVerse{R}% read next tok
  \else\ifx M\Xtok\gdef\TypeOfVerse{M}% read next tok
  \else\ifx C\Xtok\gdef\TypeOfVerse{C}%
  \else\gdef\TypeOfVerse{V}\let\next=\getnumX% continue with current
  \fi\fi\fi\next}%
%%%%  
\def\getnum#1{\futurelet\Xtok\getnumX}%
\def\getnumX{%
  \let\next=\relax%
  \ifadddigit{\Xtok}{\let\next=\getnum}%
  \ifx\next\relax% end of the number 
    \ifx:\Xtok\xdef\NumOfVerse{\the\savenum}\let\next=\endnum%
    \else% Oh, it is no verse mark, return it 
      \if V\TypeOfVerse\xdef\SAVEDTEXT{\the\savenum}%
      \else\xdef\SAVEDTEXT{\TypeOfVerse\the\savenum}%
      \fi%  
      \gdef\TypeOfVerse{X}%
      \let\next=\endnumX%
    \fi%  
  \fi\next}%
\def\endnum#1{\endnumX}% sezer pripadnou (nepovinnou) mezeru
\def\endnumX{\begintext\SAVEDTEXT}
%%%%
\def\ifadddigit#1#2{% so long as we can not get ASCII num of \nextChar
 \ONEtest0#1{#2}\ONEtest1#1{#2}\ONEtest2#1{#2}\ONEtest3#1{#2}\ONEtest4#1{#2}%
 \ONEtest5#1{#2}\ONEtest6#1{#2}\ONEtest7#1{#2}\ONEtest8#1{#2}\ONEtest9#1{#2}}%
\def\ONEtest#1#2#3{\ifx#1#2%
  \global\expandafter\savenum\expandafter{\the\savenum#1}#3\fi}%
%%%%%%%%%%%%%%%%%%%%%%%%%%%%%%%%%%%%%%%%%%%%%%%%%%%%
% "chord environment" 
% 
% chord has a form:
% [A-H][b"]?\(<modifier>\)?
% \((<general exp>)\)?\(/?<bass>\)?
% \|\(<restriced exp>\)?\(/<bass>\)?
% <modifier> .. arbitrary text, but meaningfull are: +,-,m,mi,maj,dim,sus,add
% <general exp> .. arbitraty balanced text
% <restricted exp> .. arbitrary text without slash
%
% you can write more chords separated by a space
% 
% examples:
% (A) (A") (Ab) (Am) (A7) (A6+) ... produce what you expect 
% ((A)) optional chord 
% (A((4))) optional additional 4
% (A(6/9)/E) complex exp. with bass suggestion
% (C G C) more then one chord over one nible
%
% implementation: finite automata with support for paired ()
% - all chords are recorded into allchords.aux
%%%%%%%%%%%%%%%%%%%%%%%%%%%%%%%%%%%%%%%%%%%%%%%
\newwrite\logchords%
\immediate\openout\logchords=allchords.aux
\def\logchord{%
  \immediate\write\logchords{\savechord}%
  \global\let\savechord=\empty}%
% readchord - automata kernel
\def\readchord#1{%
  \expandafter\gdef\expandafter\savechord\expandafter{\savechord#1}% 
  % hack for more chords separated by the space
  \ifx\chspace#1\logchord\readchordd{)}% empty savechord -> chord ended by space
  \else\readchordd{#1}\fi% save token and do chordnext
  \ifx\chordnext\ENDAUTOMATA \let\next=\relax% .. so stop reading 
  \else\let\next=\readchord% continue
  \fi\next}
\def\readchordd#1{%
  \ifx\CHegroupchar#1\egroup% end of pair, must have \aftergroup !!
  \else \chordnext{#1}\fi}%
%
% testtok - to test various tokens without many included "if"s
\def\testtok#1#2#3{%
  \ifx#1#2\ifx\chordnext\relax\let\chordnext=#3\fi%
  \fi}%
% main pair 
\def\beginchord{%
  \egroup% end text environment
  \setbox\lastchord=\hbox\bgroup$% 
  \catcode`\^^M=15\catcode`\(=12%
  \setchspace
  \expandafter\let\expandafter\sharptok\SharpChar
  \expandafter\let\expandafter\flattok\FlatChar
  \bgroup\aftergroup\endchord
  \bchord\readchord}%
{\catcode`\ =12\gdef\setchspace{\catcode`\ =12\let\chspace= }}%
\def\endchord{%
  \unkern$\egroup% end math chord box 
  \let\chordnext=\ENDAUTOMATA%
  \begintext}%
% one chord group
\def\bchord{%
  \bgroup\aftergroup\echord%
  \global\let\savechord=\empty%
  \let\CHegroupchar=)\let\chordnext=\bCHopt}%
\def\echord{%
  \ifx\savechord\empty% chord ended by space
    \dimen0=\fixchordspace\advance\dimen0 by \pluschordspace\hskip\dimen0%
    \let\next=\bchord%
  \else%                chord ended by ) -> end whole chord env.
    \logchord\let\next=\egroup\fi\next}%
%  
% opt. chord pair
\def\bCHopt#1{%
  \ifx(#1 \left(\bgroup\aftergroup\eCHopt%
    \let\chordnext=\bCHbase%
  \else\bCHbase{#1}%
  \fi}%
\def\eCHopt{\right)\let\chordnext=\CHerrend}%
% error if there is something else than ) after opt chord )
\def\CHerrend#1{\errmessage{Wrong end of the chord}\egroup#1}%
%
% chord base 
\def\bCHbase#1{\setformatA\MakeBase{#1}\let\chordnext=\eCHbase}%
\def\eCHbase#1{\let\chordnext=\bCHtype\CHreadtone{#1}}%
% CHreadtone - for base and bass tone modifiers
\def\CHreadtone#1{%
  \ifx#1\sharptok\MakeSharp
  \else\ifx#1\flattok\MakeFlat
  \else\chordnext{#1}%
  \fi\fi}  
%
% chord type - such text as m,mi,maj,sus,+,-,dim
\def\bCHtype#1{\CHtype{#1}}% this is now obsolate, but wait ...
\def\CHtype#1{%
  \let\chordnext=\relax%
  \testtok(#1\bCHgeneralexp% normal exp in ()
  \testtok/#1{\bCHbass\unkern}%
  \testtok+#1{\CHtype\unkern\MakeBasePlus}%
  \testtok-#1{\CHtype\unkern\MakeBaseMinus}%
  % simple exp begin with digit and end with / 
  \testtok2#1{\relax\bCHrestrictexp{#1}}% 
  \testtok4#1{\relax\bCHrestrictexp{#1}}% 
  \testtok6#1{\relax\bCHrestrictexp{#1}}% 
  \testtok7#1{\relax\bCHrestrictexp{#1}}% 
  \testtok9#1{\relax\bCHrestrictexp{#1}}% 
  % very rare cases
  \testtok1#1{\relax\bCHrestrictexp{#1}}% 
  \testtok3#1{\relax\bCHrestrictexp{#1}}% 
  \testtok5#1{\relax\bCHrestrictexp{#1}}% 
  %hack for write (A0(4)) and no (A((4)))
  \testtok0#1\bCHrestrictexp% 
  \testtok!!{\CHtype\unkern#1}% normaly type all text
}  
%
% exp pairs
\def\bCHgeneralexp#1{^\bgroup\aftergroup\eCHexp\setformatB\CHexp{#1}}%
\def\bCHrestrictexp#1{^\bgroup\aftergroup\eCHexp\setformatB
  \let\CHegroupchar=/\CHexp{#1}}%
\def\eCHexp{\MakeAfterExpKern\let\chordnext=\bCHbass}%
%
\def\CHexp#1{%
  \let\chordnext=\relax%
  \testtok+#1{\CHexp\MakeExpPlus}%
  \testtok-#1{\CHexp\MakeExpMinus}%
  \testtok(#1{\CHexp\bCHoptexp}% optional exp.
  % if we have ) here it must be end of whole chord 
  % -> override normal rules
  \testtok)#1{\relax\egroup\egroup}% end exp and main/opt
  \testtok!!{\CHexp#1}% normal type
}%
\def\bCHoptexp{(\bgroup\aftergroup\eCHoptexp\let\CHegroupchar=)}%
\def\eCHoptexp{\noexpand)}%
%
% begin bass section only if there is nontrivial material
% using _ for raise, formula for subscript raise is:
% max(d(X)+s_19,s_16,h(X)-4/5abs(s_5)), for atom X
\def\bCHbass#1{\ifx/#1 
  \else\MakeSlash%
  \fontdimen5\textfont2=2ex
  \fontdimen19\textfont2=-2ex
  \fontdimen16\textfont2=\MakeBassRaise%
  _\bgroup\aftergroup\egroup% after bass group close chord
  \textstyle\setformatC\MakeBase{#1}%
  \let\CHegroupchar=\relax% take control over ")"
  \let\chordnext=\eCHbass\fi}%
\def\eCHbass#1{\ifx)#1\egroup% end of bass and chord too
  \else\let\CHegroupchar=)\let\chordnext=\CHerrend\CHreadtone{#1}\fi}% 
%%%%%%%%%%%%%%%%%%%%%%%%%%%%%%%%%%%%%%%%%%%%%%%%%%%%%%%
%% this is the end of the parsing macros
%%%%%%%%%%%%%%%%%%%%%%%%%%%%%%%%%%%%%%%%%%%%%%%%%%%%%%%
%% Typeseting macros follows, many parameters can be changed  
%%%%%%%%%%%%%%%%%%%%%%%%%%%%%%%%%%%%%%%%%%%%%%
% typechord - put boxes \lastchord and \lasttext together
\def\fixchordspace{0.5em}  % min space between chords 
\def\pluschordspace{0.2em} % additional space, if we do prolongation of text
\def\chordraise{2.1ex}     % raise of the chord
%%%%%%%%%%%%%%%%%%%%%%%%%%%
% prolongation of the text inside of word
\def\chordinshypen{%
  \setbox0=\hbox{-}%
  \advance\dimen0 by -\wd0\divide\dimen0 by 2%
  \ifdim\dimen0<0.3em\dimen0=0.3em\else\fi%
  \hskip\dimen0{-}\hskip\dimen0%
}  
%%%%%%%%%%%%%%%%%%%%%%%%%%
% prolongation of the text in space
\def\chordinsspace{%
  \unskip%
  \ifdim\dimen0>3em{%
    \advance\dimen0 by -1.5em%
    \hskip.75em\leaders\hbox{--}\hskip\dimen0 plus 1em\hskip.75em%
  }%
  \else{\hskip\dimen0 plus 1em}%
  \fi%
}%
%%%%%%%%%%%%%%%%%%%%%%%%%%%
\def\typechord{%
  \dimen0=\wd\lastchord% d0=(d1=wd(chord)+fixchordspace)-wd(text)
  \advance\dimen0 by \fixchordspace
  \dimen1=\dimen0 
  \advance\dimen0 by -\wd\lasttext
  \raise\chordraise\rlap{\unhbox\lastchord}% insert chord with zero width
  % test if text box is just space
  \setbox1=\copy\lasttext\setbox0=\hbox{\unhbox1\unskip\unskip\unskip}%
  \ifdim\wd0=0pt\setbox\lasttext=\hbox{\hskip\dimen1}\dimen0=-1pt\fi% 
  \ifdim\dimen0<0pt{\unhbox\lasttext}% the best case 
  \else% try streach the text
    \setbox0=\hbox to \dimen1{\unhbox\lasttext}%
    \ifnum\badness<250\box0% still good case
    \else\unhbox0% fill the space or hypen a word      
       \advance\dimen0 by \pluschordspace
       \ifdim\lastskip=0.0pt\chordinshypen%
       \else\chordinsspace\fi%
    \fi%
  \fi%
}  
%%%%%%%%%%%%%%%%%%%%%%%%%%%%%%%%%%%%%%%%%%%%%
% Chord typeseting
%
%% MakeBase type base letter with possible \kern for supperscripts after
\def\MakeBase#1{\csname MakeTone#1\endcsname}
\def\MakeToneA{A\kern-.15em}
\def\MakeToneB{B}
\def\MakeToneC{C}
\def\MakeToneD{D}
\def\MakeToneE{E}
\def\MakeToneF{F}
\def\MakeToneG{G}
\def\MakeToneH{H}
%% how to type special characters
\def\MakeScrut{{\setbox0=\hbox{H}\vbox to \ht0{}}}
\def\MakeFlat{^\flat\MakeScrut}
\def\MakeSharp{^\sharp\MakeScrut}
\def\MakeSlash{\char`/}
\def\MakeExpPlus{{+}}
\def\MakeExpMinus{{-}}
\def\MakeBasePlus{\raise+.2ex\hbox{+}}
\def\MakeBaseMinus{{\scriptstyle -}}
%% some kerns and raises
\def\MakeAfterExpKern{\kern-.1em}
\def\MakeBassRaise{-.5ex}
%% set fonts for A=base tone and chord type
%%               B= exponent  C= bass tone
% useable fonts:
% qhvb - helvetica bold, PS, no sizes 
% qhvcb - helvetica condensed bold, PS, no sizes
% ebbx10 - helvetica bold
% cmbrbx10 - hlv. bld.
% ebsr10 - hlv.
% pnssb10 - pandora bld. non. sym
% t1xbss, tyxbss - hlv. bld. PS, no sizes
% vnssdc10 - round hlv. cond. bld.      
% rpagd, rphvb,rphvbrn
\newfam\ChordFam
\font\ChordFontText=qhvb at 8pt
\font\ChordFontExp=qhvb at 5pt
\font\ChordFontBass=qhvr at 8pt
\font\CHSymFont=cmmib10 at 8pt
\font\SongNameFont=csbx12 at 14pt
\font\AuthorFont=csti10
\font\TextFont=csr10 at 11pt
\font\PageFont=csbx11
%
\textfont\ChordFam=\ChordFontText
\scriptfont\ChordFam=\ChordFontExp
\scriptscriptfont\ChordFam=\ChordFontBass
\def\setformatA{\fam=\ChordFam\textstyle} 
\def\setformatB{\fam=\ChordFam\scriptstyle}
\def\setformatC{\fam=\ChordFam\scriptscriptstyle}
%
%
%%%%%%%%%%%%%%%%%%%%%%%%%%%%%%%%%%%%%%%%%%%%%%%%%%%%%
%%%%%%%%%%%%%%%%%%%%%%%%%%%%%%%%%%%%%%%%%%%%%%%%%%%%5
%% text typeseting
%%
%% how to make song header from parametres:
%% name,author,author code,source, source code
\def\MakeSongHeader#1#2#3#4#5{{%
\noindent\SongNameFont\unspace#1% song name
\if:#2:\else% author
\hskip 2em plus 1em minus 1em \AuthorFont( \unspace#2 )\fi%
\hfill\unspace#5\par%
\penalty 10000%
}}
%
%% how to begin special paragraphs
%% every paragraph in the song environment is started by
%% \noindent so the user can setup indentation of one's own
\def\MakeParBegin{%
  \if V\TypeOfVerse \hbox to \parindent {\NumOfVerse.\hfil}%
  \else\if R\TypeOfVerse \hbox to \parindent {R\NumOfVerse:\hfil}%
  \else\if C\TypeOfVerse \hbox to \parindent {Rec:\hfil}%
  \else\if M\TypeOfVerse% 
  \else\hbox{\hskip\parindent}%
  \fi\fi\fi\fi}
%
\parindent=21pt
\parskip=1ex plus 1ex minus 0.7ex
\interlinepenalty=500
%
%% this makes line indentation
\def\MakeLineIndent{%
  \if M\TypeOfVerse%
  \else\hbox{\hskip\parindent}\fi%
}  
%
\def\InterSongSkip{3ex plus 48ex}
\def\FlexibleSongSkip{3ex plus 48ex minus 1.5ex}
\def\InterSongPenalty{-50}
\def\InterVersePenalty{0}
\def\InterLinePenalty{100}
%
% page layout
\newdimen\InnerMargin\InnerMargin=0.5cm
\newdimen\OuterMargin\OuterMargin=0.7cm
\newdimen\TopMargin\TopMargin=0.7cm
\newdimen\BottomMargin\BottomMargin=0.5cm
\def\vPageSize{21cm}
\def\hPageSize{14.85cm}
%%%%%%%%%%%%%%%%%%%%%%%%%%%%%%%%%%%%%%%%%%%%
% Formating of pages - output rutine
%%%%%%%%%%%%%%%%%%%%%%%%%%%%%%%%%%%%%%%%
\def\MakePageNum{\global\advance\pagenum by 1
  \PageFont\vskip 2ex
  \hbox to \hsize{\PageFont
    \ifodd\pagenum\the\pagenum\hfill
    \else\hfill\the\pagenum\fi
  }}
\def\measure{\vbox to 0pt{\hbox to 0pt{\vrule height 18cm}}}
%
\vsize=\vPageSize 
\advance\vsize by -\TopMargin
\advance\vsize by -\BottomMargin
\newcount\pagenum
\pagenum=99
\setbox0=\vbox{\MakePageNum}
\advance\vsize by -\ht0
\pagenum=0
\vsize=2\vsize% vzdy formatuju 2 strany najednou
\voffset=-1in
\advance\voffset by \TopMargin
% horizontal setting
\hsize=\hPageSize
\advance\hsize by -\InnerMargin
\advance\hsize by -\OuterMargin
%
\output={\lyricoutput}
\def\lyricoutput{\poisepages
  \hoffset=-1in
  %left page
  {\advance\hoffset by \OuterMargin
   %\setbox1=\vbox{\hrule\hbox{\vrule\vbox{\box5\MakePageNum}\vrule}\hrule}
   \setbox1=\vbox{\hbox{\vbox{\box5\MakePageNum}}}
   \count0=\pagenum\shipout\box1}
  %right page 
  {\advance\hoffset by \InnerMargin
   %\setbox2=\vbox{\hrule\hbox{\vrule\vbox{\box6\MakePageNum}\vrule}\hrule}
   \setbox2=\vbox{\hbox{\vbox{\box6\MakePageNum}}}
   \count0=\pagenum\shipout\box2}
}
%%%%%%%%%%%%%%%%%%%%%%%%%%%%%%%%%%%%%%%%%%%%%%%%%%%%%%%%%%%%%%
% poisepages -
%   modification of original pagebreak algoritm n such way, that 
%   value of break is sum of penalty and badness of both left and right
%   page
\newbox\RightPage  
\newbox\LeftPage
\newcount\ActualBreakValue
\newskip\SaveSkip
\def\poisepages{%
  \rmvboxes % remove song vbox envelopes
  \count5=100000 % actual best page break value
  \setbox\LeftPage=\box255 % actual left page
  \setbox\LeftPage=\vbox{\unvbox\LeftPage
      \skip0=\lastskip\unskip
      \advance\skip0 by \lastskip\unskip
      \advance\skip0 by \lastskip\unskip
      \global\ActualBreakValue=\lastpenalty\unpenalty
      \global\setbox\RightPage=\vbox{}}%    
  \loop
    \setbox\LeftPage=\vbox{\unvbox\LeftPage
      \setbox0=\lastbox
      \skip0=\lastskip\unskip
      \advance\skip0 by \lastskip\unskip
      \advance\skip0 by \lastskip\unskip
      \global\SaveSkip=\skip0
      \global\ActualBreakValue=\lastpenalty\unpenalty
      \global\setbox\RightPage=\vbox{\box0\unvbox\RightPage}
    }  
%    \message{penalty: \the\ActualBreakValue }
    \setbox1=\vbox to 0.5\vsize{\unvcopy\LeftPage}\advance\ActualBreakValue by \badness
%    \message{left: \the\badness }
    \setbox2=\vbox to 0.5\vsize{\unvcopy\RightPage}\advance\ActualBreakValue by \badness
%    \message{right: \the\badness }
%    \message{act: \the\ActualBreakValue best: \the\count5 }
    \global\setbox\RightPage=\vbox{\vskip\SaveSkip\unvbox\RightPage}
    \ifnum\ActualBreakValue>\count5
    \else
      \count5=\ActualBreakValue\setbox5=\box1\setbox6=\box2
    \fi  
  \ifdim\ht\LeftPage>0pt   
  \repeat   
  \count5=0
}  
%  % \dimen0 - try vsize of the first page
%  % \dimen1 - iteration step of \dimen0
%  % \count1,2 - badness of the left,right page
%  % \count3,0 - actual,last \count1+\count2
%  %
%  \setbox0=\vbox{\unvcopy255}\dimen0=0.5\ht0% aproximate one page height
%  \ifdim\vsize<\ht0\message{Page overfull (\the\vsize, \the\ht0) !!}\fi
%  \setbox0=\box255% save copy
%  \setbox255=\copy0\splitpages% first try 
%  \ifnum\count1>\count2\dimen1=\StepSize\fi
%  \ifnum\count1<\count2\dimen1=-\StepSize\fi
%  \let\cycle=\loop
%  \loop
%     \advance\dimen0 by \dimen1
%     \count0=\count3\setbox255=\copy0\splitpages% try next height
%     \message{d: \the\dimen0, \the\dimen1;
%              b: \the\count1, \the\count2, \the\count0, \the\count3}%
%     \ifnum\count3>\count0\let\cycle=\relax\fi
%  \ifx\cycle\loop
%  \repeat
%  % revert last try
%  \setbox255=\copy0\advance\dimen0 by -\dimen1\splitpages
%  }
\def\rmvboxes{% remove song vbox envelps to allow intersong page break
  \setbox1=\vbox{\unvbox255
    \loop
      \unskip\unskip\setbox0=\lastbox
      \ifvbox0\global\setbox255=\vbox{%
        \unvbox0%
        \vskip\FlexibleSongSkip
        \penalty\InterSongPenalty%
        \unvbox255}%
    \repeat}
  \ifdim\ht1>0pt\showboxbreadth=100\showboxdepth=1\showbox1\fi}
%%
%%
%\def\splitpages{%
%%  \showboxbreadth=200\showboxdepth=1\showbox255
%  \setbox1=\vsplit255 to \dimen0
%%  \showboxbreadth=100\showboxdepth=1\showbox255
%  \setbox1=\vbox to 0.5\vsize{\unvbox1}\count1=\badness
%  \setbox2=\vbox to 0.5\vsize{\unvbox255}\count2=\badness
%  \ifnum\count1=\count2
%    \setbox3=\vbox{\unvcopy1}
%    \setbox4=\vbox{\unvcopy2}
%    \ifdim\ht3<\ht4\count1=0\count2=1
%    \else\count1=1\count2=0\fi
%  \fi  
%  \count3=\count1\advance\count3 by \count2%
%}  
% vertical setting
%




%\showboxbreadth=100
%\showboxdepth=5
%\tracingmacros=2
%\tracingcommands=2
\tracingpages=1
%\tracingrestores=1
%\tracingoutput=5

%Testovaci text s akordy
%Kaz(Abm(7+)/C#)dykonecradku(Dm(7)/Bb)je ... 
%(D+)du(Am)le(B-)zity
%
%... i prazdny
%(Cm)(C+)(C-)(Cdim)(Cb)(C#)
%(Dm)(D+)(D-)(Ddim)(Db)(D#)
%(Em)(E+)(E-)(Edim)(Eb)(E#)
%(Fm)(F+)(F-)(Fdim)(Fb)(F#)
%(Gm)(G+)(G-)(Gdim)(Gb)(G#)
%(Am)(A+)(A-)(Adim)(Ab)(A#)
%(Hm)(H+)(H-)(Hdim)(Hb)(H#)
%(Bm)(B+)(B-)(Bdim)(Bb)(B#)
%


%\inputsong{Amazonka.sng}
%\bye
