\tracingmacros=1%

\newtoks\lastchord
\let\xxxgroup=\bgroup
{%
\catcode`\@=6%
\catcode`\b=13\catcode`\#=13%
\gdef\chord@{%
  \xxxgroup%
  \catcode`\b=13\def b{\flat}%
  \catcode`\#=13\def #{\sharp}%
  \getchord%
}}%
\def\getchord#1{%
  \egroup%
  \lastchord={#1}%
  \bgroup\catcode`\ =13\gettext%
}  
{% predefinuju mezeru, ale definice bude aktivni jen az ji zapnu
\let\mezera=\ %
\catcode`\ =13%
\def {\mezera}%
}
%
\catcode`\^^M=13%
\def^^M{\break}%
% hrne text do bufferu
\newtoks\buffer
\newtoks\last
\def\gettext#1{%
  \toks0={#1}
  % testuju mezery, pokud v #1 nebo \last bude po expanzi mezera
  % tak to nebude fungovat, ale ona by tam byt nemela
  % taky nevim jak vypoustet mezery po ridicich tokenech
  \if\ #1% pokud bude v #1 mezera tak se expanduje na to same jako explic mez.
    \if\ \last%
       \let\last=\toks0%
       \toks0={}%   
  \fi\fi
  \ifx#1\chord%
     \egroup % konec predefinovane mezery
     \inschord%
     \let\next=\chord%
  \else\ifx#1^^M%
      \egroup % konec predefinovane mezery
      \inschord%
      \let\next=^^M%
    \else% let's continue
      \def\ex{\expandafter}
      \ex\ex\ex\buffer\ex\ex\ex{\ex\the\ex\buffer\the\toks0}
      \let\next=\gettext%
    \fi%
  \fi%
  \next%
}  
\catcode`\ =13
\def 

\def\inschord{%
  chord(\the\lastchord)text(\the\buffer)%
  \buffer={}\lastchord={}%
}

\chord{Am}text pisnic\chord{Dm}ky


\bye
