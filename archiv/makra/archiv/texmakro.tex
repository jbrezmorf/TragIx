% parametry algoritmu vkladani akordu nad text
\def\fixchordspace{0.5em}  % minimalni pozadovana mezera mezi akordy 
\def\pluschordspace{0.5em} % rozsireni minimalni mezery, ktere se pouzije
                           % kdyz se musi prodluzovat text
%
%                           
% typechord - parsuje akord typu A#m(7+/9)/C#
% na kus pred zavorkou (A), v zavorce (B), za lomitkem (C)
% a prida k jednotlivym castem formatovani
% vysledek je neco jako $A^{B}/C$
% navic znaky b a # jsou nahrazeny \flat resp. \sharp
%
% expsharp - parametr nesmi ani po expanzi obsahovat definice
{\catcode`\"=6%  parametry budou v definic i znaceny uvozovkami
 \catcode`\#=13%
 \gdef\expsharp"1{{\catcode`\#=13\def#{\sharp}"1}}%
}
% expflat - pozor aktivni b !!!
{\catcode`\b=13%
 \gdef\expflat#1{{\catcode`\b=13\defb{\flat}#1}}%
}
% format A,B,C - expanze jednotlivych casti akordu
\def\formatA#1{{\bf \expsharp{\expflat{#1}}}}
\def\formatB#1{^{\bf\rm #1}}
\def\formatC#1{/{\rm #1}}
% expanduje akord bez zavorek : Am/C/" nebo Am/"
\def\cutslash#1/{#1}
\def\expslash1#1/#2"{%
  \ifx"#2"\formatA{#1}%
  \else%
    \formatA{#1}%
    \formatC{\expflat{\expsharp{\cutslash{#2}}}}%
  \fi%
}
% expanduje bas po zavorce : /C#() nebo C#()
\def\expslash2#1(){%
  \formatC{\expflat{\expsharp{\ifx /#1/else #1/fi}}}%
}
% detakuje prvni zavorky, pozna umele pridane
\def\exppar#1(#2)#3"{%
  \ifx)#2)% kdyz v #2 neco bude tak se zbytek preskoci, jenom tam nesmi byt
          % \else stejne tak v #3
    \ifx"#3"\expslash1#1/"%
    \else\formatA{#1}\expslash2#3%na konci #3 je dvojice ()
    \fi%
  \else% #2 je neprazdna
    \formatA{#1}\formatB{#2}\expslash2#3%opet je tam ()  
  \fi%
}  
\def\expchord#1{\exppar#1()"}
\def\typechord#1{$\expchord$\hskip\fixchordspace}
%
% inschord#1#2 - vklada zformatovany akord #1 nad text #2
% tak aby akordy nesly pres sebe a text vypadal rozumne
% 
\def\addchord#1{\raise 2.1ex\rlap{\typechord{#1}}}
\def\chordinshypen{%
  \divide\dimen1 by 2%
  \ifdim\dimen1<0.3em\dimen1=0.3em\else\fi%
  \hskip\dimen1 plus 1em{-}\hskip\dimen1 plus 1em%
}  
\def\chordinsspace{%
  \unskip%
  \ifdim\dimen1>3em{%
    \advance\dimen1 by -1.5em%
    \hskip.75em\leaders\hbox{--}\hskip\dimen1 plus 1em\hskip.75em%
  }%
  \else{\hskip\dimen1 plus 1em}%
  \fi%
}
\def\inschord#1#2{%
   \leavevmode%
   \addchord{#1}% insert chord with zero width in whatever case
   \setbox0=\hbox{\typechord{#1}}\dimen1=\wd0% min chord width
   \setbox0=\hbox{#2}\advance\dimen1 by -\wd0% minus text width
   \showthe\dimen1
   \ifdim\dimen1<0pt{#2}% the best case 
   \else{% try streach the text
      \setbox0=\hbox to \dimen1{#2}%
      \ifnum\badness<250{\box0}% still good case
      \else{#2% fill the space or hypen a word      
         \advance\dimen1 by \pluschordspace%
         \ifdim\lastskip=0.0pt{\chordinshypen}%
         \else{\chordinsspace}%
         \fi%
      }\fi%
   }\fi%
}  

\tracingmacros=2

\chyph
\csaccents

\expflat{AbC}
\expsharp{A#C}
\inschord{Am}{Dvee}\inschord{Cmaj}{ste nas tam bylo}\par
\inschord{Dmi}{na bridge do Ciny}\par
\inschord{Emi(7/sus)/C}{I }\inschord{G}{have }

\bye

