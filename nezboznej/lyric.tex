%%%%%%%%%%%%%%%%%%%%%%%%%%%%%%%%%%%%%%%%%%%%%%%%%%%%%%%%%%%
%%  SONGBOOK style
%%  
%%  provides:
%%  \inputsong - read, parse, typeset one song file given in only parameter
%%  \makeindex - typeset an index from \jobname.toc
%%  \lyricoutput - output routine that balance songs between two pages
%%  \setlyricoutput - setup lyricoutput routine
%% 
%%
%%%%%%%%%%%%%%%%%%%%%%%%%%%%%%%%%%%%%%%%%%%%%%%%%%%%%%%%%%%%%%%%%%%%%%%%%%%%%%%%
%% CONFIGURATION
%%%%%%%%%%%%%%%%%%%%%%%%%%%%%%%%%%%%%%%%%%%%%%%%%%%%%%%%%%%%%%%%%%%%%%%%%%%%%%%
%% set fonts for A=base tone and chord type
%%               B= exponent  C= bass tone
% useable fonts:
% qhvb - helvetica bold, PS, no sizes 
% qhvcb - helvetica condensed bold, PS, no sizes
% ebbx10 - helvetica bold
% cmbrbx10 - hlv. bld.
% ebsr10 - hlv.
% pnssb10 - pandora bld. non. sym
% t1xbss, tyxbss - hlv. bld. PS, no sizes
% vnssdc10 - round hlv. cond. bld.      
% rpagd, rphvb,rphvbrn
%%%%%%%%%%%
%% FONTS
%%%%%%%%%
\newfam\ChordFam
\font\ChordFontText=qhvb at 8pt
\font\ChordFontExp=qhvb at 5pt
\font\ChordFontBass=qhvr at 8pt
\font\ChordFontExpSymb=cmsy5
\font\ChordSeparate=cmsy12 at 8pt
\font\ChordSymb=cmmib10 at 7pt %pxbmi
%
\font\SongNameFont=csbx12 at 14pt
\font\ReferenceFont=csr12 at 14pt
\font\SepDotFont=cmsy12 at 14pt
\font\AuthorFont=csti10 at 11pt
%
\font\TextFont=csr10 at 11pt
\font\PageFont=csbx11
\font\IndexFont=csr12
%
\textfont\ChordFam=\ChordFontText
\scriptfont\ChordFam=\ChordFontExp
\scriptscriptfont\ChordFam=\ChordFontBass
\scriptfont2=\ChordFontExpSymb
\textfont1=\ChordSymb
\scriptfont1=\ChordSymb
\scriptscriptfont1=\ChordSymb
\def\setformatA{\fam=\ChordFam\textstyle} 
\def\setformatB{\fam=\ChordFam\scriptstyle}
\def\setformatC{\fam=\ChordFam\scriptscriptstyle}
%%%%%%%%%%%%%%%%%%%%%%%%%
%% CHORD TYPESETTING
%%%%%%%%%%%%%%%%%%%%%%%%%
%% MakeBase type base letter with possible \kern for supperscripts after
\def\MakeBase#1{\csname MakeTone#1\endcsname}
\def\MakeToneA{A\kern-.13em}
\def\MakeToneB{B}
\def\MakeToneC{C}
\def\MakeToneD{D}
\def\MakeToneE{E}
\def\MakeToneF{F}
\def\MakeToneG{G}
\def\MakeToneH{H}
%% how to type special characters
\def\MakeScrut{{\setbox0=\hbox{H}\vbox to \ht0{}}}
\def\MakeFlat{^\flat\kern-.09em\MakeScrut}
\def\MakeSharp{^\sharp\kern-.09em\MakeScrut}
\def\MakeExpPlus{{\ChordFontExp\raise+.2ex\hbox{+}}}
\def\MakeExpMinus{{-}}
\def\MakeBasePlus{{\ChordFontText\raise+.2ex\hbox{+}}}
\def\MakeBaseMinus{{\scriptstyle -}}
%% some kerns and raises
\def\MakeAfterExpKern{\kern-.1em} % kern after exponent
\def\MakeBassRaise{-.3ex}  % raise of bass tone
\def\fixchordspace{0.4em}  % min space between chords 
\def\pluschordspace{0.3em} % additional space, if we do prolongation of text
\def\chordraise{2.2ex}     % raise of the chord
%% separation of only chords
\def\chsep{\hfil\beginchord)
  \hskip0.1em
  \raise \chordraise\hbox{{\ChordSeparate\char"6A}}
  \hbox{\hskip\fixchordspace\hskip0.1em}}
%%%%%%%%%%%%%%%%%%%%%%
%% TEXT TYPESETTING
%%%%%%%%%%%%%%%%%%%%%%
%% how to make song header from parametres:
%% name,author,author code,source, source code
\def\SeparatingDot{{\SepDotFont\char15}}
\def\MakeSongHeader#1#2#3#4#5{{%
  \noindent\SongNameFont\unspace#1% song name
  \write\TOCfile{\string\tocline{#1}{\the\pageno}}% write to toc file
  \if:#5:\else% reference
%    \hskip.2em\SeparatingDot\hskip.2em%
  { }(\unspace#5)%
  \if:#2:\else% author
  \hfill\bgroup\TextFont(\hskip.3em{\AuthorFont\unspace#2\/}\hskip.3em)\egroup
  \fi\fi\par%
  \penalty 10000%
}}
%% how to begin special paragraphs
%% every paragraph in the song environment is started by
%% \noindent so the user can setup indentation of one's own
\def\MakeParBegin{%
  \if V\TypeOfVerse \rm\hbox to \parindent {\NumOfVerse.\hfil}%
  \else\if R\TypeOfVerse \rm\hbox to \parindent {R\NumOfVerse:\hfil}%
  \else\if C\TypeOfVerse \it\hbox to \parindent {\hfil}%
  \else\if M\TypeOfVerse\rm% 
  \else\rm\hbox{\hskip\parindent}%
  \fi\fi\fi\fi}
%% this makes line indentation
\def\MakeLineIndent{%
  \if M\TypeOfVerse%
  \else\hbox{\hskip\parindent}\fi%
}
%% separation of more verses on the single line
\def\MakeVerseSeparation#1{%eat the plus sign 
  { }\egroup{}\kern-0.1em/\kern-0.1em{ }%
  \begintext}%
%% text formating parameters 
\def\InterSongSkip{4ex plus 48ex}
\def\FlexibleSongSkip{4ex plus 48ex minus 1ex}
%\def\InterSongPenalty{-50}
%\def\InterVersePenalty{0}
%\def\InterLinePenalty{100}
\def\InterSongPenalty{-10}
\def\InterVersePenalty{0}
\def\InterLinePenalty{0}
\def\ParIndent{21pt}
\def\ParSkip{1ex plus 1ex minus 0.5ex}
%%%%%%%%%%%%%%%%%%%%%%%%%%%%%%
%% PAGE LAYOUT
%%%%%%%%%%%%%%%%% 
\def\InnerMargin{0.5cm}
\def\OuterMargin{0.7cm}
\def\TopMargin{0.7cm}
\def\BottomMargin{0.5cm}
\def\vPageSize{21cm}
\def\hPageSize{14.85cm}
\def\PageLayout{%
% horizontal setting 
  \hsize=\hPageSize
  \advance\hsize by -\InnerMargin
  \advance\hsize by -\OuterMargin
  \hoffset=-1in
% vetrical setting
  \vsize=\vPageSize 
  \message{\the\vsize}
  \advance\vsize by -\TopMargin
  \advance\vsize by -\BottomMargin
  \voffset=-1in
  \advance\voffset by \TopMargin
}
%%%%%%%%%%%%%%%%%%%%%%%%%%%%%%%%%%%%%%%%%%%%%%%%%%%%%%%%%%%%%%%%%%%%%%%%%%%%%%%%%%%%%
%% MAKEINDEX
%%%%%%%%%%%%
\newwrite\TOCfile
\newdimen\colsep  % horiz. mezera mezi sloupci
\newcount\tempnum % pracovn� promenn�
\def\makeindex{
  \bgroup
  \def\cvak{\ifdim\dimen0<6pt \dimen0=9pt \else \dimen0=3pt \fi}
  \def\puntiky{\leaders\hbox to12pt{\kern\dimen0.\hss}\hfil}
  \long\def\tocline##1##2{\line{\unspace##1%
    \puntiky\hbox to1.5em{\hfil##2}}\cvak}
  \PageLayout
  \nopagenumbers
  \advance\hsize by \OuterMargin
  \advance\hsize by -\InnerMargin
  \advance\hoffset by \InnerMargin
  \baselineskip=1.2\fontdimen6\IndexFont
  \IndexFont
  \dimen0=3pt
  \colsep=1em % horiz. mezera mezi sloupci
  \splittopskip=\baselineskip
  \begmulti 2
  \input \jobname.toc %
  \endmulti 
  \vfill\break
  \immediate\openout\TOCfile=\jobname.toc
  \egroup
}
%
\def\roundtolines #1{%% zaokrouhl� na cel� n�sobky vel. r�dku
  \divide #1 by\baselineskip \multiply #1 by\baselineskip}
\def\corrsize #1{%% #1 := #1 + \splittopskip - \topskip
  \advance #1 by \splittopskip \advance #1 by-\topskip}
\def\begmulti #1 {\par\bigskip\penalty0 \def\Ncols{#1}
  \setbox0=\vbox\bgroup\penalty0
%% \hsize := ��rka sloupce = (\hsize+\colsep) / n - \colsep
  \advance\hsize by\colsep\divide\hsize by\Ncols \advance\hsize by-\colsep}
\def\endmulti{\vfil\egroup \setbox1=\vsplit0 to0pt
  %% \dimen1 := velikost zbyl�ho m�sta na str�nce
  \ifdim\pagegoal=\maxdimen \dimen1=\vsize \corrsize{\dimen1}
  \else \dimen1=\pagegoal \advance\dimen1 by-\pagetotal \fi
  \ifdim \dimen1<2\baselineskip
  \vfil\break \dimen1=\vsize \corrsize{\dimen1} \fi
  %% \dimen0 := v��ka n sloupcov� sazby po rozdelen� do sloupcu
  %% = (\ht0 + (n-1)\baselineskip) / n, zaokruhleno na r�dky
  \dimen0=\Ncols\baselineskip \advance\dimen0 by-\baselineskip
  \advance\dimen0 by \ht0 \divide\dimen0 by\Ncols
  \roundtolines{\dimen0}
  %% Rozdelit sazbu n sloupcu do str�nek nebo nerozdelit ?
  \ifdim \dimen0>\dimen1 \splitpart
  \else \makecolumns{\dimen0} \fi
  \ifvoid0 \else \errmessage{ztracen� text ve sloupc�ch?} \fi
  \bigskip}
\def\makecolumns#1{\setbox1=\hbox{}\tempnum=0
  \loop \ifnum\Ncols>\tempnum
  \setbox1=\hbox{\unhbox1 \vsplit0 to#1 \hss}
  \advance\tempnum by1
  \repeat
  \hbox{}\nobreak\vskip-\splittopskip \nointerlineskip
  \line{\unhbox1\unskip}}
\def\splitpart{\roundtolines{\dimen1}
  \makecolumns{\dimen1} \advance\dimen0 by-\dimen1
  %% \dimen0 := v��ka _zbyl�_ n sloupcov� sazby
  %% \dimen1 := pr�zdn� m�sto na str�nce = (cca) \vsize
  \vfil\break
  \dimen1=\vsize \corrsize{\dimen1}
  %% Rozdelit zbylou sazbu n sloupcu do v�ce str�nek ?
  \ifvoid0 \else
  \ifdim \dimen0>\dimen1 \splitpart
  \else \makecolumns{\dimen0} \fi \fi}
%%%%%%%%%%%%%%%%%%%%%%%%%%%%%%%%%%%%%%%%%%%%%%%%%%%%%%%%%%%%%%%%%%%%%%%%%%%%%%%%%%%%
% OUTPUT RUTINE
%%%%%%%%%%%%%%%%%%%%%%%%%%%%%%%%%%%%%%%%
% Page number typesetting
\def\MakePageNum{\vbox{%
  \global\advance\pagenum by 1%
  \PageFont\vskip 2ex%
  \hbox to \hsize{\PageFont%
    \ifodd\pagenum\the\pagenum\hfill%
    \else\hfill\the\pagenum\fi%
    }
  }}
% auxiliar rule
% \def\measure{\vbox to 0pt{\hbox to 0pt{\vrule height 18cm}}}
%
\newcount\pagenum
%%%%%%%%%%%%%%%%%%%%%%%%%%%%%%%%%%%%%%%%%%%%%%%%%%%%%%%%%%%%%%%%%%%%%%%%%%%%%
\def\setlyricoutput{%
  \PageLayout
% page numberibg
  \pagenum=99
  \setbox0=\MakePageNum
  \pagenum=0
  \message{\the\vsize,\the\ht0}
  \advance\vsize by -\ht0
  \vsize=2\vsize% vzdy formatuju 2 strany najednou
  \message{\the\vsize}
  \output={\lyricoutput}}
%%%%%%%%%%%%%%%%%%%%%%%%%%%%%%%%%%%%%%%%%%%%%%%%%%%%%%%%%%%%%%%%%%%%%%%%%
\def\lyricoutput{\poisepages
  \message{\the\vsize}
  %left page
  {\advance\hoffset by \OuterMargin
   \offinterlineskip
   %\setbox1=\vbox{\hrule\hbox{\vrule\vbox{\box5\MakePageNum}\vrule}\hrule}
   \setbox1=\vbox{\box5\MakePageNum}
   \count0=\pagenum\shipout\box1}
  %right page 
  {\advance\hoffset by \InnerMargin
   \offinterlineskip
   %\setbox2=\vbox{\hrule\hbox{\vrule\vbox{\box6\MakePageNum}\vrule}\hrule}
   \setbox2=\vbox{\box6\MakePageNum}
   \count0=\pagenum\shipout\box2}
}
%%%%%%%%%%%%%%%%%%%%%%%%%%%%%%%%%%%%%%%%%%%%%%%%%%%%%%%%%%%%%%
% poisepages -
%   modification of original pagebreak algoritm n such way, that 
%   value of break is sum of penalty and badness of both left and right
%   page
\newbox\RightPage  
\newbox\LeftPage
\newcount\ActualBreakValue
\newskip\SaveSkip
\def\poisepages{%
  \rmvboxes % remove song vbox envelopes
  \count5=100000 % actual best page break value
  \setbox\LeftPage=\box255 % actual left page
  \setbox\LeftPage=\vbox{\unvbox\LeftPage
      \skip0=\lastskip\unskip
      \advance\skip0 by \lastskip\unskip
      \advance\skip0 by \lastskip\unskip
      \global\ActualBreakValue=\lastpenalty\unpenalty
      \global\setbox\RightPage=\vbox{}}%    
  \loop
    \setbox\LeftPage=\vbox{\unvbox\LeftPage
      \setbox0=\lastbox
      \skip0=\lastskip\unskip
      \advance\skip0 by \lastskip\unskip
      \advance\skip0 by \lastskip\unskip
      \global\SaveSkip=\skip0
      \global\ActualBreakValue=\lastpenalty\unpenalty
      \global\setbox\RightPage=\vbox{\box0\unvbox\RightPage}
    }  
%    \message{penalty: \the\ActualBreakValue }
    \setbox1=\vbox to 0.5\vsize{\unvcopy\LeftPage}\advance\ActualBreakValue by \badness
%    \message{left: \the\badness }
    \setbox2=\vbox to 0.5\vsize{\unvcopy\RightPage}\advance\ActualBreakValue by \badness
%    \message{right: \the\badness }
%    \message{act: \the\ActualBreakValue best: \the\count5 }
    \global\setbox\RightPage=\vbox{\vskip\SaveSkip\unvbox\RightPage}
    \ifnum\ActualBreakValue>\count5
    \else
      \count5=\ActualBreakValue\setbox5=\box1\setbox6=\box2
    \fi  
  \ifdim\ht\LeftPage>0pt   
  \repeat   
  \count5=0
}  
%%%%%%%%%%%%%%%%%%%%%%%%%%%%%%%%%%%%%%%%%%%%%%%%%%%%%%%%%%%%%%%%%%%%%%%%%%
\def\rmvboxes{% remove song vbox envelps to allow intersong page break
  \setbox1=\vbox{\unvbox255
    \loop
      \unskip\unskip\setbox0=\lastbox
      \ifvbox0\global\setbox255=\vbox{%
        \unvbox0%
        \vskip\FlexibleSongSkip
        \penalty\InterSongPenalty%
        \unvbox255}%
    \repeat}
  \ifdim\ht1>0pt\showboxbreadth=100\showboxdepth=1\showbox1\fi}
%%
%%%%%%%%%%%%%%%%%%%%%%%%%%%%%%%%%%%%%%%%%%%%%%%%%%%%%%%%%%%%%%%%%%%%%%%%%%%%%%
%% INPUT SONG
%%%%%%%%%%%%%
\newtoks\savenum
\newbox\lasttext
\newbox\lastchord
% define some spec. tokens
\def\SharpChar{"}%
\def\FlatChar{b}%
\def\unspace{\vrule width 0pt}
%%%%%%%%%%%%%%%%%%%%%%%%%%%%%%%%%%%%%%%%%%%%%%%5
\def\inputsong#1{%
  \vbox\bgroup
  \begingroup
  \catcode`\^^M=13
  \parindent=\ParIndent
  \parskip=\ParSkip
  \interlinepenalty=\InterLinePenalty
  \expandafter\getheader\input #1 
  \ENDSONG
}
\def\ENDSONG{
  \unskip\endgroup\egroup
  \vskip\InterSongSkip
  }%
%%%%%%%%%%%%%%%%%%%%%%%%%%%%%%%%%%%%%%%%%%%%%%%%%%%%%%%%%%
%% header
\def\({(}\def\){)}
{\catcode`\(=13\catcode`\^^M=13
% getheader - reads 5 header fields, call formating macro, 
\gdef\getheader N:#1^^MA:#2^^MAC:#3^^MZ:#4^^MZC:#5^^M{%
   \MakeSongHeader{#1}{#2}{#3}{#4}{#5}%
   \beginsong}%
% beginsong - setup environment
\gdef\beginsong{%
  \def^^M{\doEOL}\let\EOLtok=^^M % to have exactly same token for testing
  \catcode`\�=0%
  \catcode`\(=13\def({\beginchord}% 
  \skipEOLs\relax}% do not eat first char
  % skip all ^^M and begin new paragraph
}%
%%%%%%%%%%%%%%%%%%%%%%%%%%%%%%%%%%%%%%%%%%%%%%%%%%%%%%
%% text reading
\def\begintext{%
  \ifvmode \noindent \TextFont\MakeParBegin \fi%
  \setbox\lasttext=\hbox\bgroup\aftergroup\endtext%
  \unspace}% to eat possible space
\def\endtext{%
  \ifvoid\lastchord\unhbox\lasttext% test trivial case
  \else\typechord%
  \fi}%
%%%%%%%%%%%%%%%%%%%%%%%%%%%%%%%%%%%%%%%%%%%%
%% line/par. govering - another finite automata with \futurelet
%% we use two \hfil at the end of line, because first could
%% be closed in the last \lasttext box, if there is a chord
%% + at beginnig of line means continuation of the previous one
%% with separation of verses
\def\doEOL{\futurelet\nextEOL\firstEOL}%
\long\def\firstEOL{%
  \ifx\nextEOL\EOLtok% end par
     \hfil\egroup\hfil\par% close text environment and par.
     \penalty\InterVersePenalty%
     \let\next=\skipEOLs% skip EOLs and begin new one
  \else\ifx\nextEOL+\let\next=\MakeVerseSeparation%
  \else\ifx\nextEOL\ENDSONG% end of line inside a paragraph
     \hfil\egroup\hfil\par
     \let\next=\relax
  \else   
    \hfil\egroup\hfil% close text environment
    \break% hard line break
    \MakeLineIndent%
    \let\next=\begintext% start new line
  \fi\fi\fi\next}%
%%%%  
\def\skipEOLs#1{\futurelet\Xtok\skipEOLsX}
\def\skipEOLsX{%
  \ifx\Xtok\EOLtok\let\next=\skipEOLs%
  \else\ifx\Xtok\ENDSONG\let\next=\relax%
  \else\let\next=\getverse%
  \fi\fi\next}%
%%%%  
\def\getverse{%
  \let\next=\getnum\global\savenum={}\xdef\SAVEDTEXT{}%
  \ifx R\Xtok\gdef\TypeOfVerse{R}% read next tok
  \else\ifx M\Xtok\gdef\TypeOfVerse{M}% read next tok
  \else\ifx C\Xtok\gdef\TypeOfVerse{C}%
  \else\gdef\TypeOfVerse{V}\let\next=\getnumX% continue with current
  \fi\fi\fi\next}%
%%%%  
\def\getnum#1{\futurelet\Xtok\getnumX}%
\def\getnumX{%
  \let\next=\relax%
  \ifadddigit{\Xtok}{\let\next=\getnum}%
  \ifx\next\relax% end of the number 
    \ifx:\Xtok\xdef\NumOfVerse{\the\savenum}\let\next=\endnum%
    \else% Oh, it is no verse mark, return it 
      \if V\TypeOfVerse\xdef\SAVEDTEXT{\the\savenum}%
      \else\xdef\SAVEDTEXT{\TypeOfVerse\the\savenum}%
      \fi%  
      \gdef\TypeOfVerse{X}%
      \let\next=\endnumX%
    \fi%  
  \fi\next}%
\def\endnum#1{\endnumX}% sezer pripadnou (nepovinnou) mezeru
\def\endnumX{\begintext\SAVEDTEXT}
%%%%
\def\ifadddigit#1#2{% so long as we can not get ASCII num of \nextChar
 \ONEtest0#1{#2}\ONEtest1#1{#2}\ONEtest2#1{#2}\ONEtest3#1{#2}\ONEtest4#1{#2}%
 \ONEtest5#1{#2}\ONEtest6#1{#2}\ONEtest7#1{#2}\ONEtest8#1{#2}\ONEtest9#1{#2}}%
\def\ONEtest#1#2#3{\ifx#1#2%
  \global\expandafter\savenum\expandafter{\the\savenum#1}#3\fi}%
%%%%%%%%%%%%%%%%%%%%%%%%%%%%%%%%%%%%%%%%%%%%%%%%%%%%
% "chord environment" 
% 
% chord has a form:
% [A-H][b"]?\(<modifier>\)?
% \((<general exp>)\)?\(/?<bass>\)?
% \|\(<restriced exp>\)?\(/<bass>\)?
% <modifier> .. arbitrary text, but meaningfull are: +,-,m,mi,maj,dim,sus,add
% <general exp> .. arbitraty balanced text
% <restricted exp> .. arbitrary text without slash
%
% you can write more chords separated by a space
% 
% examples:
% (A) (A") (Ab) (Am) (A7) (A6+) ... produce what you expect 
% ((A)) optional chord 
% (A((4))) optional additional 4
% (A(6/9)/E) complex exp. with bass suggestion
% (C G C) more then one chord over one nible
%
% implementation: finite automata with support for paired ()
% - all chords are recorded into allchords.aux
%%%%%%%%%%%%%%%%%%%%%%%%%%%%%%%%%%%%%%%%%%%%%%%
\newwrite\logchords%
\immediate\openout\logchords=allchords.aux
\def\logchord{%
  \immediate\write\logchords{\savechord}%
  \global\let\savechord=\empty}%
% readchord - automata kernel
\def\readchord#1{%
  \expandafter\gdef\expandafter\savechord\expandafter{\savechord#1}% 
  % hack for more chords separated by the space
  \ifx\chspace#1\logchord\readchordd{)}% empty savechord -> chord ended by space
  \else\readchordd{#1}\fi% save token and do chordnext
  \ifx\chordnext\ENDAUTOMATA \let\next=\relax% .. so stop reading 
  \else\let\next=\readchord% continue
  \fi\next}
\def\readchordd#1{%
  \ifx\CHegroupchar#1\egroup% end of pair, must have \aftergroup !!
  \else \chordnext{#1}\fi}%
%
% testtok - to test various tokens without many included "if"s
\def\testtok#1#2#3{%
  \ifx#1#2\ifx\chordnext\relax\let\chordnext=#3\fi%
  \fi}%
% main pair 
\def\beginchord{%
  \egroup% end text environment
  \setbox\lastchord=\hbox\bgroup$% 
  \catcode`\^^M=15\catcode`\(=12%
  \setchspace
  \expandafter\let\expandafter\sharptok\SharpChar
  \expandafter\let\expandafter\flattok\FlatChar
  \bgroup\aftergroup\endchord
  \bchord\readchord}%
{\catcode`\ =12\gdef\setchspace{\catcode`\ =12\let\chspace= }}%
\def\endchord{%
  \unkern$\egroup% end math chord box 
  \let\chordnext=\ENDAUTOMATA%
  \begintext}%
% one chord group
\def\bchord{%
  \bgroup\aftergroup\echord%
  \global\let\savechord=\empty%
  \let\CHegroupchar=)\let\chordnext=\bCHopt}%
\def\echord{%
  \ifx\savechord\empty% chord ended by space
    \dimen0=\fixchordspace\advance\dimen0 by \pluschordspace\hskip\dimen0%
    \let\next=\bchord%
  \else%                chord ended by ) -> end whole chord env.
    \logchord\let\next=\egroup\fi\next}%
%  
% opt. chord pair
\def\bCHopt#1{%
  \ifx(#1 \hbox{\ChordFontText\raise0.15ex\hbox{(}}\bgroup\aftergroup\eCHopt%
    \let\chordnext=\bCHbase%
  \else\bCHbase{#1}%
  \fi}%
\def\eCHopt{\hbox{\ChordFontText\raise0.15ex\hbox{)}}\let\chordnext=\CHerrend}%
% error if there is something else than ) after opt chord )
\def\CHerrend#1{\errmessage{Wrong end of the chord}\egroup#1}%
%
% chord base 
\def\bCHbase#1{\setformatA\MakeBase{#1}\let\chordnext=\eCHbase}%
\def\eCHbase#1{\let\chordnext=\bCHtype\CHreadtone{#1}}%
% CHreadtone - for base and bass tone modifiers
\def\CHreadtone#1{%
  \ifx#1\sharptok\MakeSharp
  \else\ifx#1\flattok\MakeFlat
  \else\chordnext{#1}%
  \fi\fi}  
%
% chord type - such text as m,mi,maj,sus,+,-,dim
\def\bCHtype#1{\CHtype{#1}}% this is now obsolate, but wait ...
\def\CHtype#1{%
  \let\chordnext=\relax%
  \testtok(#1\bCHgeneralexp% normal exp in ()
  \testtok/#1{\bCHbass\unkern}%
  \testtok+#1{\CHtype\unkern\MakeBasePlus}%
  \testtok-#1{\CHtype\unkern\MakeBaseMinus}%
  % simple exp begin with digit and end with / 
  \testtok2#1{\relax\bCHrestrictexp{#1}}% 
  \testtok4#1{\relax\bCHrestrictexp{#1}}% 
  \testtok6#1{\relax\bCHrestrictexp{#1}}% 
  \testtok7#1{\relax\bCHrestrictexp{#1}}% 
  \testtok9#1{\relax\bCHrestrictexp{#1}}% 
  % very rare cases
  \testtok1#1{\relax\bCHrestrictexp{#1}}% 
  \testtok3#1{\relax\bCHrestrictexp{#1}}% 
  \testtok5#1{\relax\bCHrestrictexp{#1}}% 
  %hack for write (A0(4)) and no (A((4)))
  \testtok0#1\bCHrestrictexp% 
  \testtok!!{\CHtype\unkern#1}% normaly type all text
}  
%
% exp pairs
\def\bCHgeneralexp#1{^\bgroup\aftergroup\eCHexp\setformatB\CHexp{#1}}%
\def\bCHrestrictexp#1{^\bgroup\aftergroup\eCHexp\setformatB
  \let\CHegroupchar=/\CHexp{#1}}%
\def\eCHexp{\MakeAfterExpKern\let\chordnext=\bCHbass}%
%
\def\CHexp#1{%
  \let\chordnext=\relax%
  \testtok+#1{\CHexp\MakeExpPlus}%
  \testtok-#1{\CHexp\MakeExpMinus}%
  \testtok(#1{\CHexp\bCHoptexp}% optional exp.
  % if we have ) here it must be end of whole chord 
  % -> override normal rules
  \testtok)#1{\relax\egroup\egroup}% end exp and main/opt
  \testtok!!{\CHexp#1}% normal type
}%
\def\bCHoptexp{(\bgroup\aftergroup\eCHoptexp\let\CHegroupchar=)}%
\def\eCHoptexp{\noexpand)}%
%
% begin bass section only if there is nontrivial material
% using _ for raise, formula for subscript raise is:
% max(d(X)+s_19,s_16,h(X)-4/5abs(s_5)), for atom X
\def\bCHbass#1{\ifx/#1 
  \else
%  \fontdimen16\textfont2=0pt
%  \fontdimen19]textfont2=0pt
  \kern-.05em/\kern-.05em
  \raise\MakeBassRaise%
  \hbox\bgroup$\bgroup\aftergroup\ebassgroup% after bass group close chord
  \setformatC\MakeBase{#1}\textstyle%
  \let\CHegroupchar=\relax% take control over ")"
  \let\chordnext=\eCHbass\fi}%
\def\eCHbass#1{\ifx)#1\egroup% end of bass and chord too
  \else\let\CHegroupchar=)\let\chordnext=\CHerrend\CHreadtone{#1}\fi}% 
\def\ebassgroup{$\egroup\egroup}
%%%%%%%%%%%%%%%%%%%%%%%%%%%%%%%%%%%%%%%%%%%%%%%%%%%%%%%
%% this is the end of the parsing macros
%% Typeseting macros follows, many parameters can be changed  
%%%%%%%%%%%%%%%%%%%%%%%%%%%%%%%%%%%%%%%%%%%%%%
% typechord - put boxes \lastchord and \lasttext together
%%%%%%%%%%%%%%%%%%%%%%%%%%%
% prolongation of the text inside of word
\def\chordinshypen{%
  \setbox0=\hbox{-}%
  \advance\dimen0 by -\wd0\divide\dimen0 by 2%
  \ifdim\dimen0<0.3em\dimen0=0.3em\else\fi%
  \hskip\dimen0{-}\hskip\dimen0%
}  
%%%%%%%%%%%%%%%%%%%%%%%%%%
% prolongation of the text in space
\def\chordinsspace{%
  \unskip%
  \ifdim\dimen0>3em{%
    \advance\dimen0 by -1.5em%
    \hskip.75em\leaders\hbox{--}\hskip\dimen0 plus 1em\hskip.75em%
  }%
  \else{\hskip\dimen0 plus 1em}%
  \fi%
}%
%%%%%%%%%%%%%%%%%%%%%%%%%%%
\def\typechord{%
  \dimen0=\wd\lastchord% d0=(d1=wd(chord)+fixchordspace)-wd(text)
  \advance\dimen0 by \fixchordspace
  \dimen1=\dimen0 
  \advance\dimen0 by -\wd\lasttext
  \raise\chordraise\rlap{\unhbox\lastchord}% insert chord with zero width
  % test if text box is just space
  \setbox1=\copy\lasttext\setbox0=\hbox{\unhbox1\unskip\unskip\unskip}%
  \ifdim\wd0=0pt\setbox\lasttext=\hbox{\hskip\dimen1}\dimen0=-1pt\fi% 
  \ifdim\dimen0<0pt{\unhbox\lasttext}% the best case 
  \else% try streach the text
    \setbox0=\hbox to \dimen1{\unhbox\lasttext}%
    \ifnum\badness<250\box0% still good case
    \else\unhbox0% fill the space or hypen a word      
       \advance\dimen0 by \pluschordspace
       \ifdim\lastskip=0.0pt\chordinshypen%
       \else\chordinsspace\fi%
    \fi%
  \fi%
}  
%%%%%%%%%%%%%%%%%%%%%%%%%%%%%%%%%%%%%%%%%%%%%%%%%%%%%%%%%%%%%%%%%%%%%%%%
%%%%%%%%%%%%%%%%%%%%%%%%%%%%%%%%%%%%%%%%%%%%%%%%%%%%%%%%%%%%%%%%%%%%%%%%
%% CHORD DIAGRAMS
%%%%%%%%%%%%%%%%%
\font\DotFont cmbsy8 scaled 1200
\font\CrossFont cmbsy8 scaled 1000
\font\PosFont cmss8
\def\StringSkip{0.2cm}
\def\FretSkip{0.4cm}
\def\LinesThick{0.01cm}
\def\ZeroBarThick{0.04cm}
\def\FingerCross{\hbox{{\CrossFont
  \setbox0=\hbox{\char2}
  \dimen0=\wd0
  \kern-0.5\dimen0\lower 0.5ex\box0\kern-0.5\dimen0}}}
\def\FingerDot{\hbox{{\DotFont
  \setbox0=\hbox{\char15}
  \dimen0=\wd0
  \kern-0.5\dimen0\lower 0.5ex\box0\kern-0.5\dimen0}}}
%\edef\FDotSkip{\box0=\hbox{\FingerDot}\wd0}
\def\barre#1#2#3{% position, from, to
    %vertical position -> d0
    \dimen0=\FretSkip\dimen1=0.5\dimen0%
    \advance\dimen0 by\LinesThick%
    \dimen0=#1\dimen0%
    \advance\dimen0 by -\dimen1%
    %start ->d1
    %length -> d2
    \dimen1=\StringSkip\advance\dimen1 by\LinesThick%
    \dimen2=#3\dimen1\dimen1=#2\dimen1%
    \advance\dimen2 by -\dimen1%
    \lower\dimen0\hbox{\hskip\dimen1\vrule height \LinesThick width \dimen2}%
    \hskip-\dimen2\hskip-\dimen1%
} 
\def\finger#1{%
  \ifx 0#1%
  \else%
  \ifx x#1%
    \dimen2=\LinesThick%
    \kern 0.5\dimen2\FingerCross\kern-0.5\dimen2%
  \else%
    \dimen0=\FretSkip\dimen1=0.5\dimen0%
    \advance\dimen0 by \LinesThick%
    \dimen0=#1\dimen0%
    \advance\dimen0 by -\dimen1%
    \dimen2=\LinesThick%
    \kern 0.5\dimen2\lower\dimen0\FingerDot\kern-0.5\dimen2%
  \fi\fi%
  \hskip\StringSkip\hskip\LinesThick%
}
\def\fingerback{
  \hskip-\StringSkip\hskip-\LinesThick%
}
\def\GridField{\vrule width \LinesThick height \FretSkip\hskip\StringSkip}
\def\GridFret{\hrule height \LinesThick%  
  \hbox{\GridField\GridField\GridField%
        \GridField\GridField \vrule width \LinesThick height \FretSkip }}
\def\Grid{%
  \setbox0=\hbox{\vtop{\GridFret\GridFret\GridFret\GridFret}}%
  \dimen0=\wd0\box0\hskip-\dimen0}
%
\def\ZeroBar{%
    \dimen0=\StringSkip\advance\dimen0 by \LinesThick%
    \dimen0=5\dimen0\advance\dimen0 by \LinesThick%
    \vbox{\hrule height \ZeroBarThick width \dimen0}%
    \hskip-\dimen0%
    }
%
\def\chdig#1#2#3{{%
  \lineskip=0pt
  \let\f=\finger\let\fb=\fingerback
  \vbox{\vfill\halign{\hfil##{\PosFont\hskip0.3em}&\hfil##\hfil\cr%
    {\PosFont\hskip1.2em}&% make space for position
    \begingroup\bgroup\TextFont\beginchord#1)\egroup\endgroup\cr%
    \noalign{{\ChordFontText\vskip0.5ex}}%
%
    \PosFont#2&%
    \dimen0=\FretSkip\dimen0=0.5\dimen0%
    \advance\dimen0 by \fontdimen5\PosFont%
    \raise\dimen0\hbox{\ifx:#2:\ZeroBar\else\fi\Grid#3}\cr%
  }}%
}}
%%%%%%%%%%%%%%%%%%%%%%%%%%%%%%%%%%%%%%%%%%%%%%%%%%%%%%%%%%%%%%%%%%%%%%555
\showboxbreadth=100
\showboxdepth=5
\tracingmacros=2
\tracingcommands=2
\tracingpages=1
\tracingrestores=1
\tracingoutput=5
%Testovaci text s akordy
%Kaz(Abm(7+)/C#)dykonecradku(Dm(7)/Bb)je ... 
%(D+)du(Am)le(B-)zity
%
%... i prazdny
%(Cm)(C+)(C-)(Cdim)(Cb)(C#)
%(Dm)(D+)(D-)(Ddim)(Db)(D#)
%(Em)(E+)(E-)(Edim)(Eb)(E#)
%(Fm)(F+)(F-)(Fdim)(Fb)(F#)
%(Gm)(G+)(G-)(Gdim)(Gb)(G#)
%(Am)(A+)(A-)(Adim)(Ab)(A#)
%(Hm)(H+)(H-)(Hdim)(Hb)(H#)
%(Bm)(B+)(B-)(Bdim)(Bb)(B#)
%
%\inputsong{Amazonka.sng}
%\bye
